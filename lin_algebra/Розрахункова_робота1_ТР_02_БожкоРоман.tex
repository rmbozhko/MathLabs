\documentclass{report}
\usepackage[english, ukrainian]{babel}
\usepackage{amsmath}
\usepackage{systeme}

%%%%%%%%%%%%%%%%%%
% Counter block
% -used to describe steps, made during equation solving

\newcounter{relctr} %% <- counter for relations
\everydisplay\expandafter{\the\everydisplay\setcounter{relctr}{0}} %% <- reset every eq
\renewcommand*\therelctr{\arabic{relctr}} %% <- label format

\newcommand\labelrel[2]{%
  \begingroup
    \refstepcounter{relctr}%
    \stackrel{\textnormal{(\arabic{relctr})}}{\mathstrut{#1}}%
    \originallabel{#2}%
  \endgroup
}
\AtBeginDocument{\let\originallabel\label} %% <- store original definition

%%%%%%%%%%%%%%%%%%

\begin{document}

\title{Лінайна алгебра і аналітична геометрія\linebreak Розрахункова робота №1}
\author{Божко Роман Вячеславович, TP-02}
\date{\today}

\maketitle

\section{Завдання №1}
\begin{equation}\begin{split}\label{eq1}
& 1)	\begin{vmatrix}
	2 & 0 & 5 \\
	0 & -1 & 10 \\
	1 & 3 & 16
	\end{vmatrix} = 2 * (-1) * 16 +- (5 * 1 * (-1) + 3 * 2 * 10) = -87 \\
& 2) \begin{vmatrix}
	2 & 0 & 5 \\
	0 & -1 & 10 \\
	1 & 3 & 16 
	\end{vmatrix}
	\begin{matrix}
	2 & 0 \\
	0 & -1 \\
	1 & 3 \\
	\end{matrix} = 2 * (-1) * 16 +- (5 * 1 * (-1) + 3 * 2 * 10) = -87 \\
& 3)	\begin{vmatrix}
	2 & 0 & 5 \\
	0 & -1 & 10 \\
	1 & 3 & 16
	\end{vmatrix} = 2 * \begin{vmatrix}-1 & 10 \\ 3 & 16\end{vmatrix} - 0 * \begin{vmatrix}0 & 5 \\ 3 & 16\end{vmatrix} + 1 * \begin{vmatrix}0 & 5 \\ -1 & 10\end{vmatrix} = 2 * (-46) + 5 = -87 \\
& 4) \begin{vmatrix}
	2 & 0 & 5 \\
	0 & -1 & 10 \\
	1 & 3 & 16
	\end{vmatrix} \labelrel={eq1:zero_forming}
	\begin{vmatrix}
	0 & -6 & -27 \\
	0 & -1 & 10 \\
	1 & 3 & 16
	\end{vmatrix} = 1 * \begin{vmatrix}-6 & -27 \\ -1 & 10\end{vmatrix} = -60 - 27 = -87
\end{split}\end{equation}
В кроці~\eqref{eq1:zero_forming} від першого рядка був віднятий 3-й рядок помножений на 2.

\section{Завдання №2}
\begin{equation}\label{eq2}\begin{split}
	& C = A^{3} + AB -7BA + 3E, A = \begin{pmatrix}-5 & -12\\ 2 & 5\end{pmatrix}, B = \begin{pmatrix}7 & 3\\2 & 1\end{pmatrix} \\
	& A^{3} = \begin{pmatrix}-5 & -12\\ 2 & 5\end{pmatrix} * \begin{pmatrix}-5 & -12\\ 2 & 5\end{pmatrix} * \begin{pmatrix}-5 & -12\\ 2 & 5\end{pmatrix} = \begin{pmatrix}-5 & -12\\ 2 & 5\end{pmatrix} \\
	& AB = \begin{pmatrix}-5 & -12\\ 2 & 5\end{pmatrix} * \begin{pmatrix}7 & 3\\2 & 1\end{pmatrix} = \begin{pmatrix}-59 & -27\\24 & 11\end{pmatrix} \\
	& BA = \begin{pmatrix}7 & 3\\2 & 1\end{pmatrix} *  \begin{pmatrix}-5 & -12\\ 2 & 5\end{pmatrix} =  \begin{pmatrix}-29 & -69\\ -8 & -19\end{pmatrix} \\
	& C = \begin{pmatrix}-5 & -12\\ 2 & 5\end{pmatrix} + \begin{pmatrix}-59 & -27\\24 & 11\end{pmatrix} - 7 * \begin{pmatrix}-29 & -69\\ -8 & -19\end{pmatrix} + 3 * \begin{pmatrix}1 & 0\\ 0 & 1\end{pmatrix} = \begin{pmatrix}142 & 444\\ 82 & 152\end{pmatrix} \\
	& C^{-1}= \frac{1}{\lvert C\rvert} \begin{pmatrix} 152 & -82 \\ -444 &  142 \end{pmatrix}^{T} = \frac{1}{-14824}\begin{pmatrix} 152 & -82 \\ -444 &  142 \end{pmatrix}^{T} = \frac{1}{-7412}\begin{pmatrix} 76 & -222 \\ -41 &  71 \end{pmatrix} \\
	& E = C^{-1}C =\frac{1}{-3706}\begin{pmatrix} 76 & -222 \\ -41 &  71 \end{pmatrix}\begin{pmatrix} 71 & 222 \\ 41 &  76 \end{pmatrix} =  \begin{pmatrix}1 & 0\\ 0 & 1\end{pmatrix}
\end{split}\end{equation}

\section{Завдання №3}
\begin{equation}\label{eq3}\begin{split}
	& \sysdelim\{.\systeme{x_1 + 2x_2 + 2x_3=3, 4x_1 - 2x_2 - 5x_3 = 5, 6x_1 - x_2 + 3x_3 = 1\quad} \to A = \begin{pmatrix}1 & 2 & 2 \\ 4 & -2 & -5 \\ 6 & -1 & 3\end{pmatrix}, B = \begin{pmatrix} 3 \\ 5 \\ 1\end{pmatrix}, X = \begin{pmatrix}x_1 \\ x_2 \\ x_3 \end{pmatrix} \\
	& 1)  \lvert A\rvert = \begin{vmatrix}1 & 2 & 2 \\ 4 & -2 & -5 \\ 6 & -1 & 3\end{vmatrix} = -79,  \lvert A_1\rvert =  \begin{vmatrix}3 & 2 & 2 \\ 5 & -2 & -5 \\ 1 & -1 & 3\end{vmatrix} = -79, \lvert A_2\rvert =  \begin{vmatrix}1 & 3 & 2 \\ 4 & 5 & -5 \\ 6 & 1 & 3\end{vmatrix} = -158, \\
& \lvert A_3\rvert =  \begin{vmatrix}1 & 2 & 3 \\ 4 & -2 & 5\\ 6 & -1 & 1\end{vmatrix} = 79, x_1 = \frac{\lvert A_1\rvert}{\lvert A\rvert} =1 , x_2 = 			\frac{\lvert A_2\rvert}{\lvert A\rvert} = 2, x_3 = \frac{\lvert A_3\rvert}{\lvert A\rvert} = -1\\
	& 2) AX = B, X = A^{-1}B = \frac{1}{-79}\begin{pmatrix}-11 & -8 & -6\\ -42 & -9 & 13\\ 8 & 13 & -10\end{pmatrix}\begin{pmatrix} 3 \\ 5 \\ 1\end{pmatrix} = \begin{pmatrix} 1 \\ 2 \\ -1\end{pmatrix} \\
	& 3)  \tilde A = \begin{pmatrix}1 & 2 & 2 & 3\\ 4 & -2 & -5 & 5\\ 6 & -1 & 3 & 1\end{pmatrix} \labelrel\sim{eq3:step_one} \begin{pmatrix}1 & 2 & 2 & 3\\ 0 & -10 & -13 & -7\\ 0 & -13 & -9 & -17\end{pmatrix} \labelrel\sim{eq3:step_two}  \begin{pmatrix}1 & 2 & 2 & 3\\ 0 & -10 & -13 & -7\\ 0 & 0 & -79 & -79\end{pmatrix} \to \\
& \sysdelim\{.\systeme{x_1 + 2x_2 + 2x_3=3, -10x_2 - 13x_3 = -7, 79x_3 = -79\quad} \to x_1 = 1, x_2 = 2, x_3 = -1 \\
& \mbox{Перевірка}: \begin{pmatrix}1 & 2 & 2 \\ 4 & -2 & -5 \\ 6 & -1 & 3\end{pmatrix}\begin{pmatrix} 1 \\ 2 \\ -1\end{pmatrix} = \begin{pmatrix} 3 \\ 5 \\ 1\end{pmatrix}
\end{split}\end{equation}
В кроці~\eqref{eq3:step_one} від 2-го рядочка було віднято перший помноженний на 4, від 3-го рядочка було віднятий перший помноженний на 6.
В кроці~\eqref{eq3:step_two} від 3-го рядочка помноженного на 10 було віднято другий помноженний на 13.

\section{Завдання №4}
\begin{equation}\label{eq4}\begin{split}
	& \lim_{x \to -1} \frac{\left( x^{2} + 3x + 2 \right)^{2}}{x^{3} + 2x^{2} - x - 2} = [\frac{0}{0}] = \lim_{x \to -1} \frac{\left((x + 1)(x + 2)\right)^{2}}{(x + 1)(x^{2} + x - 2)} \labelrel={eq4:open_brackets} \lim_{x \to -1} \frac{(x + 1)^{2}(x + 2)^{2}}{(x + 1)(x^{2} + x - 2)} = \\
	& \lim_{x \to -1} \frac{(x + 1)(x + 2)^{2}}{x^{2} + x - 2} = \lim_{x \to -1} \frac{(x + 1)(x + 2)^{2}}{(x - 1)(x + 2)} =  \lim_{x \to -1} \frac{(x + 1)(x + 2)}{x - 1} = \frac{0 * 1}{-2} = 0
\end{split}\end{equation}
В кроці~\eqref{eq4:open_brackets} відкриваємо дужки, підносячи кожен член до квадрата.

\section{Завдання №5}
\begin{equation}\label{eq5}\begin{split}
	& \sysdelim\{.\systeme{3x_1 + 2x_2 + 2x_3 + 2x_4=2, 2x_1 + 3x_2 + 2x_3 + 5x_4 = 3, 9x_1 + x_2 + 4x_3 - 5x_4 = 1, 2x_1 + 2x_2 + 3x_3 + 4x_4 = 5, 7x_1 + x_2 + 6x_3 - x_4 = 7\quad} \to \tilde A = \begin{pmatrix}3 & 2 & 2 & 2 & 2\\ 2 & 3 & 2 & 5 & 3\\ 9 & 1 & 4 & -5 & 1\\ 2 & 2 & 3 & 4 & 5\\7 & 1 & 6 & -1 & 7\end{pmatrix} \labelrel\sim{eq5:step_one} \begin{pmatrix}1 & -1 & 0 & -3 & -1\\ 2 & 3 & 2 & 5 & 3\\ 9 & 1 & 4 & -5 & 1\\ 2 & 2 & 3 & 4 & 5\\7 & 1 & 6 & -1 & 7\end{pmatrix} \labelrel\sim{eq5:step_two} \\
	&
\end{split}\end{equation}
В кроці~\eqref{eq5:conj_multp} вираз був домножений на спряжений.

\end{document}