\documentclass{report}
\usepackage[english, ukrainian]{babel}
\usepackage{amsmath}
\usepackage{systeme}
\usepackage{amssymb}

%%%%%%%%%%%%%%%%%%
% Counter block
% -used to describe steps, made during equation solving

\newcounter{relctr} %% <- counter for relations
\everydisplay\expandafter{\the\everydisplay\setcounter{relctr}{0}} %% <- reset every eq
\renewcommand*\therelctr{\arabic{relctr}} %% <- label format

\newcommand\labelrel[2]{%
  \begingroup
    \refstepcounter{relctr}%
    \stackrel{\textnormal{(\arabic{relctr})}}{\mathstrut{#1}}%
    \originallabel{#2}%
  \endgroup
}
\AtBeginDocument{\let\originallabel\label} %% <- store original definition

%%%%%%%%%%%%%%%%%%

\begin{document}
\title{Лінійна алгебра і аналітична геометрія\linebreak Розрахункова робота №1}
\author{Божко Роман Вячеславович, TP-02}
\date{\today}

\maketitle

\section{Завдання №1}
\begin{equation}\begin{split}\label{eq1}
& \mbox{а) }\begin{vmatrix}2 & 0 & 5 \\0 & -1 & 10 \\1 & 3 & 16\end{vmatrix} = 2 * (-1) * 16 - (5 * 1 * (-1) + 3 * 2 * 10) = -87 \\
& \mbox{б) }\begin{vmatrix}2 & 0 & 5 \\0 & -1 & 10 \\1 & 3 & 16\end{vmatrix}\begin{matrix}2 & 0 \\0 & -1 \\1 & 3 \\\end{matrix} = 2 * (-1) * 16 - (5 * 1 * (-1) + 3 * 2 * 10) = -87 \\
& \mbox{в) }\begin{vmatrix}2 & 0 & 5 \\0 & -1 & 10 \\1 & 3 & 16\end{vmatrix} = 2 * \begin{vmatrix}-1 & 10 \\ 3 & 16\end{vmatrix} - 0 * \begin{vmatrix}0 & 5 \\ 3 & 16\end{vmatrix} + 1 * \begin{vmatrix}0 & 5 \\ -1 & 10\end{vmatrix} = 2 * (-46) + 5 = -87 \\
& \mbox{г) }\begin{vmatrix}2 & 0 & 5 \\0 & -1 & 10 \\1 & 3 & 16\end{vmatrix} \labelrel={eq1:zero_forming} \begin{vmatrix}0 & -6 & -27 \\0 & -1 & 10 \\1 & 3 & 16\end{vmatrix} = 1 * \begin{vmatrix}-6 & -27 \\ -1 & 10\end{vmatrix} = -60 - 27 = -87
\end{split}\end{equation}
В кроці~\eqref{eq1:zero_forming} від першого рядка був віднятий 3-й рядок помножений на 2.

\section{Завдання №2}
\begin{equation}\label{eq2}\begin{split}
	& C = A^{3} + AB -7BA + 3E, A = \begin{pmatrix}-5 & -12\\ 2 & 5\end{pmatrix}, B = \begin{pmatrix}7 & 3\\2 & 1\end{pmatrix} \\
	& A^{3} = \begin{pmatrix}-5 & -12\\ 2 & 5\end{pmatrix} * \begin{pmatrix}-5 & -12\\ 2 & 5\end{pmatrix} * \begin{pmatrix}-5 & -12\\ 2 & 5\end{pmatrix} = \begin{pmatrix}-5 & -12\\ 2 & 5\end{pmatrix} \\
	& AB = \begin{pmatrix}-5 & -12\\ 2 & 5\end{pmatrix} * \begin{pmatrix}7 & 3\\2 & 1\end{pmatrix} = \begin{pmatrix}-59 & -27\\24 & 11\end{pmatrix} \\
	& BA = \begin{pmatrix}7 & 3\\2 & 1\end{pmatrix} *  \begin{pmatrix}-5 & -12\\ 2 & 5\end{pmatrix} =  \begin{pmatrix}-29 & -69\\ -8 & -19\end{pmatrix} \\
	& C = \begin{pmatrix}-5 & -12\\ 2 & 5\end{pmatrix} + \begin{pmatrix}-59 & -27\\24 & 11\end{pmatrix} - 7 * \begin{pmatrix}-29 & -69\\ -8 & -19\end{pmatrix} + 3 * \begin{pmatrix}1 & 0\\ 0 & 1\end{pmatrix} = \begin{pmatrix}142 & 444\\ 82 & 152\end{pmatrix} \\
	& C^{-1}= \frac{1}{\lvert C\rvert} \begin{pmatrix} 152 & -82 \\ -444 &  142 \end{pmatrix}^{T} = \frac{1}{-14824}\begin{pmatrix} 152 & -82 \\ -444 &  142 \end{pmatrix}^{T} = \frac{1}{-7412}\begin{pmatrix} 76 & -222 \\ -41 &  71 \end{pmatrix} \\
	& E = C^{-1}C =\frac{1}{-3706}\begin{pmatrix} 76 & -222 \\ -41 &  71 \end{pmatrix}\begin{pmatrix} 71 & 222 \\ 41 &  76 \end{pmatrix} =  \begin{pmatrix}1 & 0\\ 0 & 1\end{pmatrix}
\end{split}\end{equation}

\section{Завдання №3}
\begin{equation}\label{eq3}\begin{split}
	& \sysdelim\{.\systeme{x_1 + 2x_2 + 2x_3=3, 4x_1 - 2x_2 - 5x_3 = 5, 6x_1 - x_2 + 3x_3 = 1\quad} \to A = \begin{pmatrix}1 & 2 & 2 \\ 4 & -2 & -5 \\ 6 & -1 & 3\end{pmatrix}, B = \begin{pmatrix} 3 \\ 5 \\ 1\end{pmatrix}, X = \begin{pmatrix}x_1 \\ x_2 \\ x_3 \end{pmatrix} \\
	& 1)  \lvert A\rvert = \begin{vmatrix}1 & 2 & 2 \\ 4 & -2 & -5 \\ 6 & -1 & 3\end{vmatrix} = -79,  \lvert A_1\rvert =  \begin{vmatrix}3 & 2 & 2 \\ 5 & -2 & -5 \\ 1 & -1 & 3\end{vmatrix} = -79, \lvert A_2\rvert =  \begin{vmatrix}1 & 3 & 2 \\ 4 & 5 & -5 \\ 6 & 1 & 3\end{vmatrix} = -158, \\
& \lvert A_3\rvert =  \begin{vmatrix}1 & 2 & 3 \\ 4 & -2 & 5\\ 6 & -1 & 1\end{vmatrix} = 79, x_1 = \frac{\lvert A_1\rvert}{\lvert A\rvert} =1 , x_2 = 			\frac{\lvert A_2\rvert}{\lvert A\rvert} = 2, x_3 = \frac{\lvert A_3\rvert}{\lvert A\rvert} = -1\\
	& 2) AX = B, X = A^{-1}B = \frac{1}{-79}\begin{pmatrix}-11 & -8 & -6\\ -42 & -9 & 13\\ 8 & 13 & -10\end{pmatrix}\begin{pmatrix} 3 \\ 5 \\ 1\end{pmatrix} = \begin{pmatrix} 1 \\ 2 \\ -1\end{pmatrix} \\
	& 3)  \tilde A = \begin{pmatrix}1 & 2 & 2 & 3\\ 4 & -2 & -5 & 5\\ 6 & -1 & 3 & 1\end{pmatrix} \labelrel\sim{eq3:step_one} \begin{pmatrix}1 & 2 & 2 & 3\\ 0 & -10 & -13 & -7\\ 0 & -13 & -9 & -17\end{pmatrix} \labelrel\sim{eq3:step_two}  \begin{pmatrix}1 & 2 & 2 & 3\\ 0 & -10 & -13 & -7\\ 0 & 0 & -79 & -79\end{pmatrix} \to \\
& \sysdelim\{.\systeme{x_1 + 2x_2 + 2x_3=3, -10x_2 - 13x_3 = -7, 79x_3 = -79\quad} \to x_1 = 1, x_2 = 2, x_3 = -1 \\
& \mbox{Перевірка}: \begin{pmatrix}1 & 2 & 2 \\ 4 & -2 & -5 \\ 6 & -1 & 3\end{pmatrix}\begin{pmatrix} 1 \\ 2 \\ -1\end{pmatrix} = \begin{pmatrix} 3 \\ 5 \\ 1\end{pmatrix} = B
\end{split}\end{equation}
В кроці~\eqref{eq3:step_one} від 2-го рядочка було віднято перший помноженний на 4, від 3-го рядочка було віднятий перший помноженний на 6.
В кроці~\eqref{eq3:step_two} від 3-го рядочка помноженного на 10 було віднято другий помноженний на 13.

\section{Завдання №4}
\begin{equation}\label{eq4}\begin{split}
	& A =\begin{pmatrix}2 & -1\\0 & 5\end{pmatrix}, B =\begin{pmatrix}-3 & 4\\5 & 2\end{pmatrix}, C =\begin{pmatrix}1 & 8\\4 & -7\end{pmatrix}, D =\begin{pmatrix}1 & -3\\-4 & 0\end{pmatrix} \\
	& \mbox{a) }BCX = 2D + A, BC = F = \begin{pmatrix}13 & -52\\13 & 26\end{pmatrix}, 2D + A = G = \begin{pmatrix}4 & -7\\-8 & 5\end{pmatrix}  \\
& FX = G, X = F^{-1}G, F^{-1} = \frac{1}{1014}\begin{pmatrix}26 & -13\\52 & 13\end{pmatrix}^{T} = \frac{13}{1014}\begin{pmatrix}2 & 4\\-1 & 1\end{pmatrix} \\
& X = F^{-1}G = \frac{13}{1014}\begin{pmatrix}2 & 4\\-1 & 1\end{pmatrix}\begin{pmatrix}4 & -7\\-8 & 5\end{pmatrix} = \frac{13}{1014}\begin{pmatrix}-24  & 6\\-12 & 12\end{pmatrix} =  \frac{1}{39}\begin{pmatrix}-4  & 1\\-2 & 2\end{pmatrix}\\
	& \mbox{б) }DXA=B-C, B-C=F=\begin{pmatrix}-4 & -4\\1 & 9\end{pmatrix}, X = D^{-1}FA^{-1} \\
& D^{-1}F = G = \frac{1}{-12}\begin{pmatrix}0 & 3\\4 & 1\end{pmatrix}\begin{pmatrix}-4 & -4\\1 & 9\end{pmatrix} = \frac{1}{-12}\begin{pmatrix}3 & 27\\-15 & -7\end{pmatrix} \\
& X = GA^{-1} = \frac{1}{-12}\begin{pmatrix}3 & 27\\-15 & -7\end{pmatrix} * \frac{1}{10}\begin{pmatrix}5 & 1\\0 & 2\end{pmatrix} = \frac{1}{-120}\begin{pmatrix}15 & 57\\-75 & -29\end{pmatrix}
\end{split}\end{equation}

\section{Завдання №5}
\begin{equation}\label{eq5}\begin{split}
	& \sysdelim\{.\systeme{3x_1 + 2x_2 + 2x_3 + 2x_4=2, 2x_1 + 3x_2 + 2x_3 + 5x_4 = 3, 9x_1 + x_2 + 4x_3 - 5x_4 = 1, 2x_1 + 2x_2 + 3x_3 + 4x_4 = 5, 7x_1 + x_2 + 6x_3 - x_4 = 7\quad} \to \tilde A = \begin{pmatrix}3 & 2 & 2 & 2 & 2\\ 2 & 3 & 2 & 5 & 3\\ 9 & 1 & 4 & -5 & 1\\ 2 & 2 & 3 & 4 & 5\\7 & 1 & 6 & -1 & 7\end{pmatrix} \labelrel\sim{eq5:step_one} \begin{pmatrix}1 & -1 & 0 & -3 & -1\\ 2 & 3 & 2 & 5 & 3\\ 9 & 1 & 4 & -5 & 1\\ 2 & 2 & 3 & 4 & 5\\7 & 1 & 6 & -1 & 7\end{pmatrix} \labelrel\sim{eq5:step_two} \\
	& \begin{pmatrix}1 & -1 & 0 & -3 & -1\\ 0 & 5 & 2 & 11 & 5\\ 0 & 10 & 4 & 22 & 10\\ 0 & 4 & 3 & 10 & 7\\0 & 8 & 6 & 20 & 14\end{pmatrix} \labelrel\sim{eq5:step_three} \begin{pmatrix}1 & -1 & 0 & -3 & -1\\0 & 5 & 2 & 11 & 5\\ 0 & 4 & 3 & 10 & 7\end{pmatrix} \labelrel\sim{eq5:step_four} \begin{pmatrix}1 & -1 & 0 & -3 & -1\\ 0 & 1 & -1 & 1 & -2\\ 0 & 4 & 3 & 10 & 7\end{pmatrix} \labelrel\sim{eq5:step_five} \\
	& \begin{pmatrix}1 & -1 & 0 & -3 & -1\\ 0 & 1 & -1 & 1 & -2\\ 0 & 0 & 7 & 6 & 15\end{pmatrix} \labelrel\to{eq5:rang_check} \lvert A\rvert = \begin{vmatrix} 1 & -1 & 0\\ 0 & 1 & -1\\ 0 & 0 & 7\end{vmatrix} = 7 \to rang(A) = 3, \\
	& \lvert \tilde A\rvert = \begin{vmatrix} -1 & -1 & 0\\ -2 & 1 & -1\\ 15 & 0 & 7\end{vmatrix} = -6 \to rang(\tilde A) = 3 \to rang(A) = rang(\tilde A) \neq n\mbox{, де n - к-лсть змінних} \labelrel\to{eq5:capelli_theorem}\\
	& \sysdelim\{.\systeme{x_1 - x_2 - 3x_4=-1, x_2 - x_3 + x_4 = -2, 7x_3 + 6x_4 = 15\quad} \labelrel\to{eq5:vars_type} \\
	& \sysdelim\{.\systeme{x_1 = -3 + \frac{1}{7}(15 - 6x_4) - x_4 + 3x_4, x_2 = -2 + \frac{1}{7}(15 - 6x_4) - x_4, x_3 = \frac{1}{7}(15 - 6x_4), x_4 = C\mbox{, }C\in\mathbb{R}\quad} \\
	& \mbox{Відповідь: } x_1 = \frac{(-6 + 8C)}{7}, x_2 = \frac{(1 - 13C)}{7}, x_3 = \frac{(15 - 6C)}{7}, x_4 = C\mbox{, }C\in\mathbb{R}
\end{split}\end{equation}
В кроці~\eqref{eq5:step_one} від 1-го рядочка було віднято другий рядок.
В кроці~\eqref{eq5:step_two} від 2-го рядочка було віднято перший помноженний на 2. Від 3-го рядочка було віднято перший помноженний на 9. Від 4-го рядочка було віднято перший помноженний на 2. Від 5-го рядочка було віднято перший помноженний на 7.
В кроці~\eqref{eq5:step_three} з матриці були вилучені лінійно залежні рядки.
В кроці~\eqref{eq5:step_four} від 2-го рядочка було віднято третій.
В кроці~\eqref{eq5:step_five} від 3-го рядочка було віднято другий помноженний на 4.
В кроці~\eqref{eq5:rang_check} початкова матриця була зведена до трикутного виду. Знайдемо визначники матриці коефіцієнтів і розширеної матриці для того, аби встановити ранги цих матриць і визначити сумісність/несумісність системи.
В кроці~\eqref{eq5:capelli_theorem} була використана теорема Кронекера-Капелі, а саме: СЛАР має безліч рішень, оскільки ранг матриць не дорівнює кількості змінних.
В кроці~\eqref{eq5:vars_type} залежні змінні були виражені через вільні.
\end{document}