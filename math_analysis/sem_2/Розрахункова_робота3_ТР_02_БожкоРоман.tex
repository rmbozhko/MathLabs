\documentclass{report}
\usepackage[english, ukrainian]{babel}
\usepackage{amsmath}
\usepackage{systeme}

\newcounter{relctr} %% <- counter for relations
\everydisplay\expandafter{\the\everydisplay\setcounter{relctr}{0}} %% <- reset every eq
\renewcommand*\therelctr{\alph{relctr}} %% <- label format

\newcommand\labelrel[2]{%
  \begingroup
    \refstepcounter{relctr}%
    \stackrel{\textnormal{(\alph{relctr})}}{\mathstrut{#1}}%
    \originallabel{#2}%
  \endgroup
}
\AtBeginDocument{\let\originallabel\label} %% <- store original definition

\begin{document}

\title{Математичний аналіз II\linebreakРозрахункова робота №1}
\author{Божко Роман Вячеславович, TP-02}
\date{\today}

\maketitle

\section{Завдання №1}
\begin{equation}\label{eq1}\end{equation}

\section{Завдання №2}
\begin{equation}\label{eq2}\begin{split}
	\int \frac{x^{2}}{x^{6} - 1} dx = \begin{vmatrix}t = x^{3}\\dt = 3x^{2}dx\\\frac{dt}{3} = x^{2}dx\end{vmatrix} = \frac{1}{3}\int\frac{dt}{t^2 - 1} = \frac{1}{3} \cdot \frac{1}{2} \ln\left(\frac{t - 1}{t + 1}\right) + C = \frac{1}{6} \ln\left(\frac{x^3 - 1}{x^3 + 1}\right) + C
\end{split}\end{equation}

\section{Завдання №3}
\begin{equation}\label{eq3}\begin{split}
	\int \sin 2x \cdot e^{\cos 2x} dx = \begin{vmatrix}t = \cos 2x\\dt = -2\sin2xdx\\-\frac{dt}{2} = \sin2xdx\end{vmatrix} = -\frac{1}{2} \int e^{t} dt = -\frac{1}{2} e^{\cos 2x} + C
\end{split}\end{equation}

\section{Завдання №4}
\begin{equation}\label{eq4}\begin{split}
	& \int\frac{x - 1}{\sqrt{5 + 2x - x^2}}dx = \int \frac{x - 1}{\sqrt{-(x^2 - 2x - 5)}}dx = \begin{vmatrix}t = x^2 - 2x - 5\\dt = (2x - 2)dx\\\frac{dt}{2} = (x - 1)dx\end{vmatrix} = \frac{1}{2}\int\frac{dt}{\sqrt{-t}} =\\
	& -\frac{1}{2} \cdot 2\sqrt{-t} + C = - \sqrt{5 + 2x - x^2} + C
\end{split}\end{equation}

\section{Завдання №5}
\begin{equation}\label{eq5}\begin{split}
	 \int (10x - 3)(5x^2 - 3x + 8)^8 dx = \begin{vmatrix}t = 5x^2 - 3x + 8\\dt = (10x - 3)dx\end{vmatrix} = \int t^8 dt = \frac{t^9}{9} + C = \frac{1}{9}(5x^2 - 3x + 8)^9 + C
\end{split}\end{equation}

\section{Завдання №6}
\begin{equation}\label{eq6}\begin{split}
	& \int \frac{dx}{-3x^2 + 2x - 5} = -\frac{1}{3}\int\frac{dx}{x^2 -\frac{2x}{3} + \frac{5}{3}} = -\frac{1}{3}\int\frac{dx}{x^2 - 2 \cdot x \cdot \frac{1}{3} + \frac{1}{9} - \frac{1}{9} + \frac{5}{3}} =\\
	& -\frac{1}{3}\int\frac{dx}{\left(x - \frac{1}{3}\right)^2 + \left(\frac{\sqrt{14}}{3}\right)^2} = -\sqrt{14}\arctg\left(\frac{3x-1}{\sqrt{14}}\right) + C
\end{split}
\end{equation}

\section{Завдання №7}
\begin{equation}\label{eq7}
	\lim_{x \to 0} \frac{\tg x - x}{\sin x - x^{2}} = \left[\frac{0}{0}\right] \overset{L}{=} \lim_{x \to 0} \frac{\frac{1}{\cos^{2}x} - 1}{\cos x - 2x} = 0
\end{equation}

\section{Завдання №8}
\begin{equation}\label{eq8}\begin{split}
	&y = x - x^{3}, x_0 = -1\\
	&\mbox{Рівняння дотичної до графіка функції: } y - f(x_0) = f'(x_0)(x - x_0)\\
	& f'(x) = 1 - 3x^{2}, f'(x_0) = -2, f(x_0) = 0, y =  -2x - 2\\
	&\mbox{Рівняння нормалі до графіка функції: } y - f(x_0) = -\frac{1}{f'(x_0)}(x - x_0)\\
	& y - 0 = -\frac{1}{-2}(x + 1), y = \frac{x + 1}{2}
\end{split}
\end{equation}

\section{Завдання №9}
\begin{equation}\label{eq9}\begin{split}
	& \int\ln(x^2 + 1) dx = \begin{vmatrix}u =\ln(x^2 + 1), du = \frac{2xdx}{x^2 + 1}\\dv=dx, v = \int dx = x\end{vmatrix} = x\ln(x^2 + 1) - \int \frac{2x^2 dx}{x^2 + 1} =\\
	& x\ln(x^2 + 1) - 2\int\frac{x^2 + 1 - 1}{x^2 + 1} dx = x\ln(x^2 + 1) - 2\left(\int dx - \int \frac{dx}{x^2 + 1}\right) =\\
	& x\ln(x^2 + 1) - 2(x - \arctg x + C) = x(\ln(x^2 + 1) - 2) + 2\arctg x + C
\end{split}\end{equation}

\section{Завдання №10}
\begin{equation}\begin{split}
	& \int(x-2)\sin(\frac{x}{2})dx = \begin{vmatrix}u = x - 2, du = dx\\dv = \sin(\frac{x}{2})dx, v = \int \sin(\frac{x}{2})dx = 2\int \sin(\frac{x}{2})d(\frac{x}{2}) = -2\cos(\frac{x}{2})\end{vmatrix} =\\
	& -2\cos(\frac{x}{2})(x - 2) - \int -2\cos(\frac{x}{2})dx = -2\cos(\frac{x}{2})(x - 2) + 2\int\cos(\frac{x}{2})dx =\\
	&  -2\cos(\frac{x}{2})(x - 2) + 4\int\cos(\frac{x}{2})d(\frac{x}{2}) = 4\sin(\frac{x}{2}) - 2\cos(\frac{x}{2})(x - 2) + C
\end{split}\end{equation}

\section{Завдання №11}
\begin{equation}\begin{split}
	& \int x^2 \arctg(3x)dx
\end{split}\end{equation}

\section{Завдання №12}
\begin{equation}\begin{split}
	& \int \frac{6x^2 + 3x - 15}{(x^2 + 4x + 3)(x - 2))}dx = \begin{vmatrix}x_{1, 2} = \frac{-4 \pm \sqrt{16 - 4 \cdot 1 \cdot 3}}{2} = \frac{-4 \pm 2}{2} = \begin{cases}-1\\-3\end{cases}\end{vmatrix} =\\
	& \int \frac{6x^2 + 3x - 15}{(x - 2)(x + 1)(x + 3)}dx = \ast\\
	& \frac{A}{x-2} + \frac{B}{x+1} + \frac{C}{x +3} = \frac{6x^2 + 3x - 15}{(x - 2)(x + 1)(x + 3)}\\
	& A(x + 1)(x + 3) + B(x - 2)(x + 3) + C(x - 2)(x + 1) = 6x^2 + 3x - 15\\
	& A(x^2 + 4x + 3) + B(x^2 + x - 6) + C(x^2 - x - 2) = 6x^2 + 3x - 15\\
	& \sysdelim\{.\systeme{A + B + C = 6, 4A + B - C = 3, 3A - 6B - 2C = -15\quad} \to \begin{pmatrix}1 & 1 & 1 & 6\\ 4 & 1 & -1 & 3\\ 3 & -6 & -2 & -15\end{pmatrix} \sim \begin{pmatrix}1 & 1 & 1 & 6\\ 0 & -3 & -5 & -21\\ 0 & -9 & -5 & -33\end{pmatrix} \sim\\
	& \begin{pmatrix}1 & 1 & 1 & 6\\ 0 & -3 & -5 & -21\\ 0 & 0 & 10 & 30\end{pmatrix} \to \sysdelim\{.\systeme{A + B + C = 6, (-3)B - 5C = -21, 10C = 30\quad} \to \sysdelim\{.\systeme{A = 1, B = 2, C = 3\quad}\\
	& \ast = \int(\frac{1}{x-2} + \frac{2}{x+1} + \frac{3}{x + 3})dx = \int\frac{dx}{x-2} + 2\int\frac{dx}{x+1} + 3\int\frac{dx}{x + 3} =\\
	& \ln(x - 2) + 2\ln(x + 1) + 3\ln(x + 3) + C 
\end{split}\end{equation}

\section{Завдання №19}
\begin{equation}\begin{split}
	& \int \frac{x^2 dx}{x\sqrt{x^2 - 3}} = \int \frac{xdx}{\sqrt{x^2 - 3}} = \begin{vmatrix}t = x^2 - 3\\dt = 2xdx\\xdx = \frac{dt}{2}\end{vmatrix} = \frac{1}{2}\int\frac{dt}{\sqrt{t}} = \frac{1}{2}\cdot2\sqrt{t} + C = \sqrt{x^2 - 3} + C
\end{split}\end{equation}

\section{Завдання №21}
\begin{equation}\begin{split}
	& \int \cos(2x)\sin x dx=\frac{1}{2} \left(\int \sin(-x)dx + \int\sin(3x)dx)\right) = \frac{1}{2}\left(-\int\sin(x)dx + \frac{1}{3}\int\sin(3x)d(3x)\right) =\\
	& \frac{1}{2}\cos x - \frac{1}{6} \cos(3x) + C
\end{split}\end{equation}

\section{Завдання №22}
\begin{equation}\begin{split}
	& \int \sin^2(5x)dx = \frac{1}{2}\int(1 - \cos (10x))dx = \frac{1}{2}(\int dx - \int\cos(10x)dx) =\\
	& \frac{1}{2}(x - \frac{1}{10}\int\cos(10x)d(10x)) = \frac{1}{2}(x - \frac{1}{10}\sin(10x)) + C
\end{split}\end{equation}

\section{Завдання №23}
\begin{equation}\begin{split}
	& \int \cos^3(3x)dx = \int \cos^2(3x)\cos(3x)dx = \frac{1}{2}\int(1+\cos(6x))\cos(3x)dx =\\
	& \frac{1}{2}\int(\cos(3x) + \cos(3x)\cos(6x))dx = \frac{1}{2}\int\cos(3x)dx + \frac{1}{2}\int(\cos(-3x) + \cos(9x))dx =\\
	& \frac{1}{2}(\frac{1}{3}\int\cos(3x)d(3x) + \frac{1}{2}(\frac{1}{3}\int\cos(3x)d(3x) + \frac{1}{9}\int\cos(9x)d(9x)) =\\
	& \frac{1}{2}(\frac{1}{3}\sin(3x) + \frac{1}{2}(\frac{1}{3}\sin(3x) + \frac{1}{9}\sin(9x)) + C = \frac{1}{4}\sin(3x) + \frac{1}{36}\sin(9x) + C
\end{split}\end{equation}

\section{Завдання №24}
\begin{equation}\begin{split}
	& \int \sin^4(x)dx = \int \sin^2(x)\sin^2(x)dx = \frac{1}{4}\int(1 - \cos(2x))(1 - \cos(2x))dx =\\
	& \frac{1}{4}\int(1 - 2\cos(2x) + \cos^2(2x))dx = \frac{1}{4}(\int dx - \int\cos(2x)d(2x) + \frac{1}{2}\int(1 + \cos(4x)))dx =\\
	& \frac{1}{4}(\int dx - \int\cos(2x)d(2x) + \frac{1}{2}(\int dx + \frac{1}{4}\int\cos(4x)d(4x))) =\\
	& \frac{1}{4}(x - \sin(2x) + \frac{1}{2}(x + \frac{1}{4}\sin(4x))) + C = \frac{3}{8}x - \frac{1}{4}\sin(2x) + \frac{1}{32}\sin(4x) + C
\end{split}\end{equation}

\section{Завдання №29}
\begin{equation}\begin{split}
	& \int\sqrt{x^2 + 3x - 4}dx = \int\sqrt{x^2 + 2\cdot x\cdot\frac{3}{2} + \frac{9}{4} - \frac{9}{4} - \frac{16}{4}}dx = \int(\sqrt{(x + \frac{3}{2})^2 - (\frac{5}{2})^2}) dx =\\
	& \int(\sqrt{(x + \frac{3}{2})^2 - (\frac{5}{2})^2}) d(x+\frac{3}{2}) = \frac{x + \frac{3}{2}}{2}(\sqrt{(x + \frac{3}{2})^2 - (\frac{5}{2})^2}) - \frac{25}{8}\ln\left((x + \frac{3}{2}) + \sqrt{(x + \frac{3}{2})^2 - (\frac{5}{2})^2}\right) + C
\end{split}\end{equation}

\section{Завдання №30}
\begin{equation}\begin{split}
	& \int\sqrt{3 - 4x - x^2}dx = \int\sqrt{7 - (x + 2)^2}dx = \int\sqrt{(\sqrt{7})^2 - (x + 2)^2}dx =\\
	& \frac{x + 2}{2}\sqrt{7 - (x + 2)^2} + \frac{7}{2}\arcsin\left(\frac{x + 2}{\sqrt{7}}\right) + C
\end{split}\end{equation}
\end{document}