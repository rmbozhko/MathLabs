\documentclass{report}
\usepackage[english, ukrainian]{babel}
\usepackage{amsmath}

\newcounter{relctr} %% <- counter for relations
\everydisplay\expandafter{\the\everydisplay\setcounter{relctr}{0}} %% <- reset every eq
\renewcommand*\therelctr{\alph{relctr}} %% <- label format

\newcommand\labelrel[2]{%
  \begingroup
    \refstepcounter{relctr}%
    \stackrel{\textnormal{(\alph{relctr})}}{\mathstrut{#1}}%
    \originallabel{#2}%
  \endgroup
}
\AtBeginDocument{\let\originallabel\label} %% <- store original definition

\begin{document}

\title{Математичний аналіз I\linebreakРозрахункова робота №1}
\author{Божко Роман Вячеславович, TP-02}
\date{\today}

\maketitle

\section{Завдання №1}
\begin{equation}\begin{split}\label{eq1}
	& \lim_{n \to \infty} \frac{(3 - n)^{4} - (2 - n)^{4}}{(1 - n)^{3} - (1 + n)^{3}} = [\frac{\infty - \infty}{\infty - \infty}] \labelrel={eq1:cubic_fact} \\
	& \lim_{n \to \infty} \frac{(3 - n)^{4} - (2 - n)^{4}}{((1 - n) - (1 + n))((1 - n)^{2} + (1 - n)(1 + n) + (1 + n)^{2})} = \\
	& \lim_{n \to \infty} \frac{(3 - n)^{4} - (2 - n)^{4}}{-2n((1 - n)^{2} + (1 - n)(1 + n) + (1 + n)^{2})} \labelrel={eq1:four_pow_fact} \\
	& -\frac{1}{2} \lim_{n \to \infty} \frac{((3 - n) - (2 - n))((3 - n) + (2 - n))((3 - n)^{2} + (2 - n)^{2})}{n((1 - n)^{2} + (1 - n)(1 + n) + (1 + n)^{2})} = \\
	& -\frac{1}{2} \lim_{n \to \infty} \frac{(5 - 2n)((3 - n)^{2} + (2 - n)^{2})}{n((1 - n)^{2} + (1 - n)(1 + n) + (1 + n)^{2})} = \\
	& -\frac{1}{2} \lim_{n \to \infty} \frac{(5 - 2n)(2n^{2} - 10n + 13)}{n((1 - n)^{2} + 1 - n^{2} + (1 + n)^{2})} = \frac{1}{2} \lim_{n \to \infty} \frac{4n^{3} - 30n^{2} + 76n - 65}{n(n^{2} + 3)} = \\
	& \frac{1}{2} \lim_{n \to \infty} \frac{n^{3}(4 - \frac{30}{n} + \frac{76}{n^{2}} - \frac{65}{n^{3}})}{n(n^{2} + 3)} = \frac{1}{2} \lim_{n \to \infty} \frac{n^{3}(4 - \frac{30}{n} + \frac{76}{n^{2}} - \frac{65}{n^{3}})}{n^{3} + 3n} = \frac{1}{2} \lim_{n \to \infty} \frac{n^{3}(4 - \frac{30}{n} + \frac{76}{n^{2}} - \frac{65}{n^{3}})}{n^{3}(1 + \frac{3}{n^{2}})} = \\
	&  \frac{1}{2} \lim_{n \to \infty} \frac{4 - 0 + 0 - 0}{1 + 0} = \frac{1}{2} * 4 = 2
\end{split}\end{equation}
В кроці~\eqref{eq1:cubic_fact} була використана формула скороченого множення для різниці кубів.
В кроці~\eqref{eq1:four_pow_fact} була використана формула скороченого множення для різниці виразів четвертої степені.

\section{Завдання №2}
\begin{equation}\label{eq2}\begin{split}
	& \lim_{n \to \infty} (n - \sqrt[3]{n^{3} - 5})n\sqrt{n} = [\infty - \infty] \labelrel={eq2:cubic_fact} \lim_{n \to \infty} \frac{(n - \sqrt[3]{n^{3} - 5})(n\sqrt{n})(n^{2} + n\sqrt[3]{n^{3} - 5} + (\sqrt[3]{n^{3} - 5})^{2})}{(n^{2} + n\sqrt[3]{n^{3} - 5} + (\sqrt[3]{n^{3} - 5})^{2})}  = \\ 
	& \lim_{n \to \infty} \frac{(n^{3} - n^{3} + 5)(n\sqrt{n})}{(n^{2} + n\sqrt[3]{n^{3} - 5} + (\sqrt[3]{ n^{3} - 5})^{2})} = \lim_{n \to \infty}			\frac{5n^{\frac{3}{2}}}{n^{2} + n\sqrt[3]{n^{3}(1 - \frac{5}{n^{3}})} + (\sqrt[3]{n^{3}(1 - \frac{5}{n^{3}})})^{2}} \labelrel={eq2:min_eq} \\
	& \lim_{n \to \infty} \frac{5n^{\frac{3}{2}}}{3n^{2}} = \frac{5}{3} \lim_{n \to \infty} \frac{\sqrt{n}}{n} = \frac{5}{3} \lim_{n \to \infty} \frac{1}{n\sqrt{n}} = 0
\end{split}\end{equation}
В кроці~\eqref{eq2:cubic_fact} була використана формула скороченого множення для різниці кубів.
В кроці~\eqref{eq2:min_eq} були відкинуті елементи знаменника, котрі прямують до 0.

\section{Завдання №3}
\begin{equation}\label{eq3}
	\lim_{n \to \infty} \left(\frac{n^{2} - 1}{n^{2}} \right)^{n^{4}} = \lim_{n \to \infty} \left(1 - \frac{1}{n^{2}} \right)^{n^{4}} = [1^{\infty}] = \lim_{n \to \infty} \left(\left(1 + (\frac{1}{-n^{2}}) \right)^{-n^{2}} \right)^{\frac{n^{4}}{-n^{2}}} \labelrel={eq3:2_wond_lim} \lim_{n \to \infty} e^{-n^{2}} = 0
\end{equation}
В кроці~\eqref{eq3:2_wond_lim} була використана друга чудова границя.


\section{Завдання №4}
\begin{equation}\label{eq4}\begin{split}
	& \lim_{x \to -1} \frac{\left( x^{2} + 3x + 2 \right)^{2}}{x^{3} + 2x^{2} - x - 2} = [\frac{0}{0}] = \lim_{x \to -1} \frac{\left((x + 1)(x + 2)\right)^{2}}{(x + 1)(x^{2} + x - 2)} \labelrel={eq4:open_brackets} \lim_{x \to -1} \frac{(x + 1)^{2}(x + 2)^{2}}{(x + 1)(x^{2} + x - 2)} = \\
	& \lim_{x \to -1} \frac{(x + 1)(x + 2)^{2}}{x^{2} + x - 2} = \lim_{x \to -1} \frac{(x + 1)(x + 2)^{2}}{(x - 1)(x + 2)} =  \lim_{x \to -1} \frac{(x + 1)(x + 2)}{x - 1} = \frac{0 * 1}{-2} = 0
\end{split}\end{equation}
В кроці~\eqref{eq4:open_brackets} відкриваємо дужки, підносячи кожен член до квадрата.

\section{Завдання №5}
\begin{equation}\label{eq5}\begin{split}
	& \lim_{x \to 1} \frac{\sqrt{x - 1}}{\sqrt[3]{x^{2} - 1}} = [\frac{0}{0}] \labelrel={eq5:conj_multp}\lim_{x \to 1} \frac{(\sqrt{x - 1})(\sqrt{x + 1})}{(\sqrt{x + 1})(\sqrt[3]{x^{2} - 1)}} = \lim_{x \to 1} \frac{\sqrt{x^{2} - 1}}{(\sqrt{x + 1})(\sqrt[3]{x^{2} - 1)}} = \\
	& \lim_{x \to 1} (x^{2} - 1)^{\frac{1}{2}}(x^{2} - 1)^{-\frac{1}{3}}(x + 1)^{-\frac{1}{2}} =  \lim_{x \to 1} (x^{2}-1)^{\frac{1}{6}}(x + 1)^{-\frac{1}{2}} = \lim_{x \to 1} \frac{\sqrt[6]{x^{2} - 1}}{\sqrt{x+1}} = \frac{0}{2} = 0
\end{split}\end{equation}
В кроці~\eqref{eq5:conj_multp} вираз був домножений на спряжений.

\section{Завдання №6}
\begin{equation}\label{eq6}\begin{split}
	& \lim_{x \to 0} \frac{3x^{2} - 5x}{\sin3x} = [\frac{0}{0}] = \lim_{x \to 0} \frac{3x(3x^{2} - 5x)}{3x\sin3x} \labelrel={eq6:1_wond_lim} \lim_{x \to 0} \frac{3x^{2} - 5x}{3x} = \\
	& \lim_{x \to 0} \frac{3x - 5}{3} =  \lim_{x \to 0} \frac{0 - 5}{3} = -\frac{5}{3}
\end{split}
\end{equation}
В кроці~\eqref{eq6:1_wond_lim} була використана перша чудова границя.

\section{Завдання №7}
\begin{equation}\label{eq7}\begin{split}
	& \lim_{x \to \pi} \frac{1 + \cos3x}{\sin^{2}7x} = [\frac{0}{0}] \labelrel={eq7:var_replacement} \lim_{t \to 0} \frac{1 + \cos3(t + \pi)}{\sin^{2}7(t + \pi)} = \lim_{t \to 0} \frac{1 + \cos(3t + 3\pi)}{\sin^{2}(7t + 7\pi)} \labelrel={eq7:trig_add} \\
	& \lim_{t \to 0} \frac{1 + \cos3t\cos3\pi - \sin3t\sin3\pi}{(\sin7t\cos7\pi + \sin7\pi\cos7t)^{2}} \labelrel={eq7:elem_elimination_1} \lim_{t \to 0} \frac{1 - \cos3t}{(-\sin7t)^{2}} \labelrel={eq7:cos_double_arg} \lim_{t \to 0} \frac{2\sin^{2}\frac{3t}{2}}{(-\sin7t)^{2}} \labelrel={eq7:sin_double_arg}  \\
	& \lim_{t \to 0} \frac{2\sin^{2}\frac{3t}{2}}{(-2\sin\frac{7t}{2}\cos\frac{7t}{2})^{2}}\labelrel={eq7:elem_elimination_2}  \lim_{t \to 0} \frac{2\sin^{2}\frac{3t}{2}}{4\sin^{2}\frac{7t}{2}} = \frac{1}{2} \lim_{t \to 0} \frac{\sin\frac{3t}{2} * \sin\frac{3t}{2}}{\sin\frac{7t}{2} * \sin\frac{7t}{2}} \labelrel={eq7:lim_equival} \\
	& \frac{1}{2} \lim_{t \to 0} \frac{9t^{2}}{4} * \frac{4}{49t^{2}} = \frac{9}{98}
\end{split}
\end{equation}
В кроці~\eqref{eq7:var_replacement} була використана заміна: $t = x - \pi; t \to 0; x = t + \pi$ .\linebreak
В кроці~\eqref{eq7:trig_add} була використана формула додавання для тригонометричних функцій: $\cos(\alpha + \beta) = \cos\alpha\cos\beta-\sin\alpha\sin\beta$.
В кроці~\eqref{eq7:elem_elimination_1} вирази були спрощені за допомогою членів, котрі прямують до остаточних значень: $\cos3\pi, \cos7\pi \to -1; \sin3\pi, \sin7\pi \to 0$.
В кроці~\eqref{eq7:cos_double_arg} була використана формула тригонометричної функції подвійного аргументу: $\cos2\alpha = 1 - 2\sin^{2}\alpha$.
В кроці~\eqref{eq7:sin_double_arg} була використана формула тригонометричної функції подвійного аргументу: $\sin2\alpha = 2\sin\alpha\cos\alpha$.
В кроці~\eqref{eq7:elem_elimination_2} була використана формула скороченого множення для різниці кубів.
В кроці~\eqref{eq7:lim_equival} була використана еквівалентність: $\sin x \sim x$.


\section{Завдання №8}
\begin{equation}\label{eq8}\begin{split}
	& \lim_{x \to 0} \frac{6^{2x} - 7^{-2x}}{\sin3x - 2x} = [\frac{0}{0}] = \lim_{x \to 0} \frac{6^{2x} - 7^{-2x}}{2x(\frac{\sin3x}{2x} - 1)} = \lim_{x \to 0} \frac{6^{2x} - 7^{-2x}}{2x(\frac{3\sin3x}{2 * 3x} - 1)}  \labelrel={eq8:1_wond_lim} \\
	& \lim_{x \to 0} \frac{6^{2x} - 7^{-2x}}{2x*(\frac{1}{2})} = \lim_{x \to 0} \frac{6^{2x} - 1 - (7^{-2x} - 1)}{x} \labelrel={eq8:lim_split} \lim_{x \to 0} \frac{6^{2x} - 1}{x} - \frac{7^{-2x} - 1}{x} =\\
	& \lim_{x \to 0} \frac{2(6^{2x} - 1)}{2x} - \frac{(-2)(7^{-2x} - 1)}{-2x} \labelrel={eq8:pow_wond_lim} 2\ln6 + 2\ln7 = \ln42^{2} =\ln(1764)
\end{split}
\end{equation}
В кроці~\eqref{eq8:1_wond_lim} була використана перша чудова границя.
В кроці~\eqref{eq8:lim_split} була використана властивість арифметики границь, а саме, що границя різниці двох функцій дорівнює різниці границь цих функцій.
В кроці~\eqref{eq8:pow_wond_lim} була використана чудова границя: $\lim_{\alpha \to 0} \frac{b^{\alpha} - 1}{\alpha} = \ln b$.


\section{Завдання №9}
\begin{equation}\label{eq9}
	\lim_{x \to -1} \frac{x^{3} + 1}{\sin(x + 1)} = [\frac{0}{0}] \labelrel={eq9:cubic_fact} \lim_{x \to -1} \frac{(x + 1)(x^{2} - x + 1)}{\sin(x + 1)} \labelrel={eq9:1_wond_lim} \lim_{x \to -1} x^{2} - x + 1 = 3
\end{equation}
В кроці~\eqref{eq9:cubic_fact} була використана формула скороченого множення для суми кубів.
В кроці~\eqref{eq9:1_wond_lim} була використана перша чудова границя.

\section{Завдання №10}
\begin{equation}
	\lim_{x \to 0} \left( \frac{\sin4x}{x} \right)^{\frac{2}{(x+2)}} = [\frac{0}{0}] = \lim_{x \to 0} \left( \frac{4\sin4x}{4x} \right)^{\frac{2}{(x+2)}} \labelrel={eq10:1_wond_lim} \lim_{x \to 0} 4^{\frac{2}{(x+2)}} = 4
\end{equation}
В кроці~\eqref{eq10:1_wond_lim} була використана перша чудова границя.

\end{document}